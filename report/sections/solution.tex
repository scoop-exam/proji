\section{Problem Solution}
\label{sec:solution}

In this section we describe our proposed solution for the Santa Claus problem using the SCOOP Eiffel framework.

\subsection{Design Assumptions}
\label{subsec:design}

Before going into the implementation details, let's clarify some of the design
assumptions we took. Some of them have been taken just to better represent the
actual scenario, but could be easily relaxed or avoided at all.

\begin{itemize}
    \item we assume a) the total number of elves in the system b) the
total number of reideers in the system and c) the number of waiting elves that
makes santa wake up and help them configurable and not limited by
the problem specification.
    \item an elf is implemented as an iterative process: each iteration, or
\textit{step}, represents the action of building a new toy and
every time an elf builds a toy there is a random possibility to end up in a
failure, which in turn will lead the elf to ask for Santa's help. Moreover, an elf
has a fixed maximum number of failures, after which he simply stops building
toys. This assumption has been took just to obtain a more realistic behavior
and can be easily relaxed. Having $10$ iterative elves with 3 as the maximum allowed
number of failures is the same of having $30$ single-iteration elves --- in terms of
number of requests to Santa.
    \item a reindeer is implemented as an iterative process too. In this case
the iteration is just a random chance of going to Santa and the lifecycle ends
when it actually happens.
    \item in order to increase the non-determinism and to make the concurrency
more evident, for each elf and for each reindeer we introduce random waiting
times at the beginning and at the end of their lifecycles.
\end{itemize}

\subsection{Solution Description}
\label{subsec:sol}

\img{class_diagram}{The class diagram for Santa Claus problem.}

Figure \ref{fig:class_diagram} shows the UML class diagram generated from the
source code using Eiffel Studio.
According to the problem specification and to our design assumptions, each step
of the \texttt{ELF}'s lifecycle is a sequence of 3 calls to \texttt{SANTA}: the
\texttt{ELF}'s \texttt{go\_to\_santa}, \texttt{get\_help} and
\texttt{come\_back} commands. The sequence of calls is performed only if there is a failure in
building the toy during the current step.

As for the \texttt{REINDEER}, each step of its lifecycle is a sequence of 2
calls to \texttt{SANTA}: the \texttt{REINDEER}'s \texttt{go\_to\_santa} and
\texttt{get\_hitched} commands. The sequence is performed only if the \texttt{REINDEER}
decides to come back during the current step.

The class \texttt{SANTA} represents the shared resource between elves and reindeers.
It keeps track of the system's status with its boolean attributes and it acts
as the system's logger. Having a centralized logger allows us to get simple and effective traces of execution.

\texttt{SANTA} relevant features are categorized into 3 groups:

\begin{itemize}
    \item \textbf{Elf-accessible commands:} an ELF calls \texttt{enqueue\_elf}
when it \texttt{go\_to\_santa} and \texttt{dequeue\_elf} when it
\texttt{come\_back}. 
    \item \textbf{Reindeer-accessible commands}: a REINDEER calls
\texttt{enqueue\_reindeer} when it \texttt{go\_to\_santa} and hitch when it ask
to \texttt{get\_hitched}.
    \item \textbf{Internal behavior}: internally SANTA \texttt{help\_elves}
when the maximum number of waiting elves (\texttt{max\_elves}) is reached and
\texttt{prepare\_sleigh} when every reindeer has come.
\end{itemize}

\texttt{SANTA}'s internal state is represented by the following boolean attributes:

\begin{itemize}
    \item \texttt{is\_busy}: tells if \texttt{SANTA} is busy in helping some elves.
    \item \texttt{is\_ready}: tells if \texttt{SANTA} is ready to hitch reindeers.
    \item \texttt{is\_xmas}: tells if it is Christmas and SANTA has left --- everything is over.
\end{itemize}

\img{state_diagram}{The state diagram for Santa.}

Figure \ref{fig:state_diagram} shows the state diagram for \texttt{SANTA} class.
Every node of the graph is obtained associating a truth value to the variables above
as specified in table \ref{table:santa_truth_table}.

\begin{table}[h!]
\centering
\caption{Truth table to obtain Santa's possible states.}
\label{table:santa_truth_table}
\begin{tabular}{|c|c|c|c|}
\hline
\textbf{is\_busy} & \textbf{is\_ready} & \textbf{is\_xmas} & \textbf{State}           \\ \hline
0                 & 0                  & 0                 & Sleeping                 \\ \hline
0                 & 0                  & 1                 & //                       \\ \hline
0                 & 1                  & 0                 & Hitching reindeers       \\ \hline
0                 & 1                  & 1                 & Christmas                \\ \hline
1                 & 0                  & 0                 & Helping elves (waiting)  \\ \hline
1                 & 0                  & 1                 & //                       \\ \hline
1                 & 1                  & 0                 & Helping elves (hitching) \\ \hline
1                 & 1                  & 1                 & Forced help              \\ \hline
\end{tabular}
\end{table}

Santa start in ``Sleeping'' state; every time $3$ elves come at his door he
wakes up and helps them (i.e., ``Helping elves (waiting for reindeers)'').
When the $9th$ reindeers comes, Santa starts hitching reindeers (``Hitching
reindeers''). Even in this state, elves can come and Santa's behavior is the
same (i.e., he helps them in batches of $3$ --- ``Helping elves (hitching)'').
When all the reindeers have been hitched.  Santa declares that ``Christmas''
has come; he helps the last elves that are outside of his door (``Forced
help'') and he start bringing awesome presents to children.

\subsection{The Main Issues}

We believe that we faced two main difficulties while solving the Santa Claus problem:

\begin{enumerate}
    \item There is a single shared resource that is accessed with different
purposes by two distinct entities.
    \item The shared resource can become unavailable at any point in time,
potentially leaving some of its clients (e.g. the elves) in a pending
condition.
\end{enumerate}

For the first one, it was not that difficult to properly model Santa in order
to manage the application logic of both the elves and the reindeers thanks to
SCOOP's exclusive access guarantees.

It was more complicated to handle Santa's unavailability issue. When it is
Christmas, it is ensured by problem specification that all the reindeers have
been hitched (that is, there is no \texttt{REINDEER} still running). Moreover
the termination condition of an \texttt{ELF} is that is has to be Christmas.
The problem is that elf starvation could happen when \texttt{SANTA} leaves
while there are some elves that are in the middle of their life --- i.e., the
execution of the last step of their lifecycle. For instance, it could happen
that \texttt{SANTA} has left but there are still 2 running elves that are both
enqueued (e.g.  arrived at \texttt{SANTA}) and waiting for the third elf to
arrive and to wake up \texttt{SANTA}. It could be that, actually, there is no
other running \texttt{ELF}, so the 2 aforementioned elves will wait forever.

In order to solve this issue we decided to adopt a simple policy: when it is
Christmas, \texttt{SANTA} forces the helping of all the currently running elves
before leaving. So, as soon as an \texttt{ELF} arrives at Santa's, he/she gets
also helped. \texttt{SANTA} has basically two different ways of handling elves:
before and after Christmas has come, and we ensure that when it is Christmas
and \texttt{SANTA} leaves, the running process terminates.

The source code and all the resources for our solution to the Santa Claus
problem can be found in our public repository\footnote{\url{https://github.com/scoop-exam/proji}}.
An example output of the program can be found in the appendices at subsection
\ref{subsec:trace}.
